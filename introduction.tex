Services accessed from mobile devices is increasingly provided hosted in the cloud. Migrating services to the cloud compliments and virtualizes a mobile devices resources. Nevertheless, the delay as a result of executing code or storing resources fully or partially in the cloud introduces and unwanted delay that can disrupt the desired seamless user experience.

Cloud services are traditionally hosted in aggregated data centres scattered throughout the globe. Theirs ability to cost effectively host services is in general fundamentally derived by their size and energy efficient. 

The delay introduced when externalising a service to a data centre is a product of the geographic distance to the data centre, the congestion on the intermediate core network, the mobile access network, and the performance of the data centre.

As the number of cloud service increase and the number of devices rely on cloud services increase 
so will the congestion in the access network and thus also delay. Moreover, as more traditional hardware or device local service are virtualized to the cloud, the demand to low latency response will increase. Conceivably, storage can be fully virtualized, critical control processes can be migrated to the cloud. In the advent of the emergence the internet of things, data from an vast number of sensors, actuators, and peripheral interaction points will flood the internet with traffic, ranging as vastly in size and QoS needs.

In order to be able to reduce latency and be able to formulate relevant SLAs the service hosting nodes will need to reduce their geographic proximity to the consumption device or process.

We refer to proposed paradigm of migrating cloud service hosting and execution closer to the consumption device as the \xcloud. The distribution of cloud data centre hardware and the virtualizetion of resources can proposedly coexist with future virtualized mobile networks.