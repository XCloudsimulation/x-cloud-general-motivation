\subsection{Performance and mobility}
\begin{table}[H]
\small
\begin{tabular}{|l|l|l|l|c|} \hline
\textbf{Conference} & \textbf{Date} & \textbf{Loc.} & \textbf{Deadline} & \textbf{Pages} \\ \hline
CloudCom & Dec 15-18 & SG & Jul 31 & 8 \\ \hline
BDCS & Jan 26-27 & SG & Aug 29 & 10	\\ \hline
6th CCW & Dec 8-11 & LDN & Jul 25 & 8 \\ \hline
\end{tabular}
\end{table}

\subsubsection{Abstract}
In an \xcloud topology the resources are dispersed throughout the network. Given that services strictly migrate with the user from , depending on the placement of the service hosts, user mobility will affect the performance perceived performance of the service. The model takes into account the effort of migrating a service and the service performance degradation it introduces.

This paper determines the service performance in relation to the placement of the \xcloud host nodes and explores the user and provider utility of subscribing to an \xcloud node at a certain network depth.

\subsubsection{Related research}


\subsubsection{Desired model}
As the the topology of the \xcloud and future mobile networks is yet to be determined, and due to the fact that we want to research the effects of mobility without a socio-economic model, the network 

\subsubsection{Simulation}
\begin{itemize}
\item \textbf{Service}\\
The traffic generated by and the usage pattern of a simple web application is characteristic of any smaller mobile application. The HTTP traffic model in \cite{liu2001traffic} is 

\item \textbf{Mobility}\\
The 2 dimensional, multi model, mobility model \cite{bettstetter2001smooth} provides a 
\item \textbf{Mobile Access Network}\\

\item \textbf{Core network}\\

\item \textbf{Server}\\

\end{itemize}