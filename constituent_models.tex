\subsubsection{Data Centre}
Cloud model : \cite{5959161}
Web serve model : \cite{1191656}

\subsubsection{Radio access network}
Because the mobile access network is a service access qualifier, the mechanisms of the network is relatively irrelevant to the primary research topics. The network can appropriately be modelled with a series of delays. 

It is a constituent objective of our research to determine relevant X-Cloud/NGN \footnote{Next Generation Network} symbiotic topologies. These topolgies will be feed into the simulation model and will conceivably encompass, resource placement and dimension, cell sizes, and radio resource provisioning \cite{kwan2010mobility,racz2007handover,salo2010practical}.

Alternatively we can deploy a full scale LTE system simulation tool such as LTE-Sim \cite{5634134}. However, such a solution would leave us with a skewed level of detail across the entire model set. The core and mobile networks would have a proportionally greater level of detail than the serve model. The granularity of the model might be incompatible with 

Our basic research topics will require a homogeneous, equidistant, and equirange cell topology. Although it is reasonable to assume that future networks hosting an X-Cloud will be distributed. 

\subsubsection{Base station}
The base station can conceivably be modelled with a queue and a delay proportional to its propagation distance to its associated "C-RAN" node.

\subsubsection{Core network}
The essential property of the core network is bandwidth and delay. Both of which can be modelled with queues.

\begin{itemize}
\item Latency
\begin{itemize}
\item "Point-to-Point" core network delay model \cite{choi2007analysis}
\item "One-hop" core network router queue delay model  \cite{papagiannaki2003measurement}
\end{itemize}
\end{itemize}

\subsubsection{Mobility}
A smooth random walk, unobstructed, bounded, edge-aware mobility model will provide a uniformly distributed dispersion of users across the simulation domain \cite{Bettstetter:2001:SBS:381591.381600}. The model is two-dimensional and provides pedestrian, bicycle, and auto mobile mobility modes. The model is uniform and does thus not take into account any socio-demographic variations, and local clusters. Nevertheless, exploring specific demographic and urban settings is beyond the scope of our basic research topics. Furthermore, in the absence of a socio-demographic and urban scenarios, an aggregate mobility mode will be deployed.

\subsubsection{Service}
There is a multitude of appropriate service models. 

\begin{itemize}
\item Light weight 1-tiere web service model from 1998 \cite{barford1998generating}
\item Modern light weight 1-tiere web service model \cite{lee2007new}
\item YouTube workload generator \cite{Gill:2007:YTC:1298306.1298310}
\item 3-tiered open-loop web service model \cite{1521145}
\item Web browsing behaviour : \cite{liu2010understanding}
\item Cloud service usage patterns : \cite{zhao2009cloud}
\end{itemize}