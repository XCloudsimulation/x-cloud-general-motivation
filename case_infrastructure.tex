Distributed virtualized mobile networks will rely on centralized compute nodes for higher level link management. One node will proposedly host multiple base stations, to which they connect over a network link, much like the Ericsson Radio Dot System \cite{ericsson_dot}, but at a larger scale. The size of these virtualization resource nodes is proportional to the maximum distance they can reside from the radio nodes, given the induced propagation delay. Supposedly these virtualization resource nodes will the placed in the vicinity of the core IP network. The virtualization resource nodes can be seen as to define geographic areas whos boundaries are defined by the reach of the mobile network which it serves. Depending on the level of desired provision and load balancing flexibility, these geographic domains will overlap to varying degrees.

The virtualization resource nodes are conceivably constructed of generic x86 or ARM servers, hosting VMs or containers within which the virtualized mobile network infrastructure is executed. Given the placement of the virtualization resource node, any free or designually excess capacity can be used hist other services.

The topology is designed to optimize the use of radio resources, the geographic domains which the virtualization resource nodes constitute do not necessarily overlap or map the demographic area which \xcloud services operate.