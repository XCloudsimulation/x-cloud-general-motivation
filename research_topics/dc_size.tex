\subsection{Size of edge data centres (number of CPU, memory, etc.) \emph{Ericsson}}
\subsubsection{Proposed research}

The size of an edge data center is primarily bounded by two factors: the costs of infrastructure and maintenance as well as the availability of power supply.
On the one hand, the economy of scale indicates the minimum size.
Creating many tiny edge data centers is simply unprofitable.
To set up even the smallest data center one has to provide computing resources, power supply, network connection, cooling system, building, etc.
On the other hand, the maximum size of the edge data center is determined by the available power supply.
Data centers need a lot of energy which is mostly consumed by cooling system.
However, nowadays antennas and base stations can be localized in places where providing such power supply is difficult (e.g. roofs of residential blocks or towers of historical buildings).

In our research we would like to determine the optimal size of edge data centers.
We will take into account such factors as:
\begin{itemize}
\item costs of infrastructure and maintenance
\item demography
\item available power supply
\end{itemize}

\subsubsection{Related research}
Current research focuses on the cost of running big data centers with particular emphasis on network \cite{Greenberg:2008:CCR:1496091.1496103}, cooling \cite{Pakbaznia:2009:MDC:1594233.1594268}.
There are also cost models for planning, development and operation of a data center available \cite{patel2005cost}.
According to our knowledge there is no research about the optimal size of edge data centers.