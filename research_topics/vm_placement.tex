\subsection{VM placement and migration decisions \emph{Umeå}}
\subsubsection{Proposed research}
In the \xcloud a service will either exists only locally or distributed, but purposely serves a geographically and demographically bounded populous. The units on which the service is hosted are of limited capacity and cannot he universally virtualized as a traditional data center, with relatively unlimited capacity over time. Give the load on each of the nodes and their geographic relevance to the demographic populous they server, services will conceivably need to be mobile, migrating, dispersing, and contracting to minimize such properties as cost, load, and traffic, also taking into account the cost of the migration itself.

We propose research into an appropriate centralized or distributed, load balancing cost function, taking into account :

\begin{itemize}
\item Incurred migration load on host and receiver
\item Incurred migration induced network congestion
\item Network congestion
\item Delay \/ RTT to client to aggregate client base. In other words, minimize delay to aggregate delay/latency to all its served clients. \emph{Perhaps separate research topic preceding this one}
\item Client mobility
\end{itemize}

A property such as energy consumption is perhaps irrelevant to a topology of small data centres as domain of movement is fairly limited and bounded by the demographic service, the rate of which a service migrates is to some extent bounded by the mobility of its users. Nevertheless, energy can become a relevant parameter if the service is allowed to migrate between the \xcloud and a traditional data centre or if the energy profiles of the local \xcloud hosts, is heterogeneous. As the serviced domain is bounds each service by its sociogeographic profile and the fact that latency gains are fairly small accounting for thermal emissions and reuse would be counter productive, and should be dealt with optimally by each node independently. When optimizing for latency, inherent thermal efficiency will conceivably seldom correlate with the service sociogeographic domain. 

\subsubsection{Related research}
Existing research in this area is mainly focuses on load balancing and provisioning between larger distributed data centres with static users.